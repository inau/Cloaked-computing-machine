\chapter{Banker's Algorithm}
This chapter is about Banker's algorithm. It's about testing the safety by simulating the allocation of resources for several processes/threads.

\section{Requirements}
\begin{itemize}
\item Implement Banker's algorithm
\begin{itemize}
\item Check if state is secure
\item Allocate and release resources for threads
\end{itemize}
\end{itemize}

\section{Problem}
The problem we try to solve is that resource allocation with more than one thread or process can end up deadlocking. For instance when one thread has allocated some of its needed resources but not all, and another thread has allocated the resources the first one needs, but it needs the resources allocated by the first one, both of them will wait indefinately until the other one releases the resources.

\section{Solution}
A solution to the problem is Banker's algorithm, which is a way to manage resource allocation. If done correctly, deadlocks will not happen. Also we implemented a security method that checks if the current state is safe.\\

These two things together allow threads to allocate and release resources safely and keeping the state safe while doing so.\\

We found some algorithms in the book(Operating System Concepts 9th edition P. 327-328), which we use for allocating resources and checking the safety of the state.

\subsection{Resource allocation algorithm}
Allocating resources is done in three steps, if we follow the algorithm on page 328 in the book mentioned above.\\

Given a request vector Request\textsubscript{i} for the process P\textsubscript{i}, Process P\textsubscript{i} wants \textit{k} instances of resource type R\textsubscript{k} if Request\textsubscript{i}[\textit{j}] == \textit{k}.\\

When a request for resources is made by process P\textsubscript{i}, the following actions will be taken:

\subsubsection{Step 1:}
If the process is requesting more resources than it needs, raise an error condition, else go to step 2. Request\textsubscript{i} $<=$ Need\textsubscript{i}.

\subsubsection{Step 2:}
If the process is requesting more resources than are available, the process will have to wait, since there are no more resources. If resources are available, go to step 3.

\subsubsection{Step 3:}
Allocate the resources the process is asking for, by modifying the state:
\begin{align*}
Available    &= Available-Request_i\\
Allocation_i &= Allocation_i + Request_i\\
Need_i       &= Need_i - Request_i
\end{align*}
After allocation the state should be safe, but a check is nessecary. If the check fails then the allocation should be undone and the function should raise an error condition. If it is safe, the allocation was completed.

\lstinputlisting[xleftmargin=40.0px, language=C, numbers=left, linerange={111-167}, firstnumber=111, breaklines=true, showspaces=false, showstringspaces=false]{code/banker.c}

\subsection{State safety algorithm}
To ensure the system's safety, we implement Banker's safety algorithm, which checks the current state for errors and reports back. This algorithm is also described in the book(Operating System Concepts 9th edition P. 327-328) with 4 steps, which we will explain below:

\subsubsection{Step 1:}
First we create two new vectors; \textit{Work} and \textit{Finish}, where \textit{Work}'s length is the number of columns (resources) and \textit{Finish}' length is the number of rows (processes).\\

We then initialize \textit{Work} to be the same as the state's \textit{Available} vector and set all \textit{Finish}' values to false (here 0).\\

In order for the state to be deemed safe, we go through some checks, which if safe, set the Finish value for the safe process just checked to true (here 1). The next steps consists of the checks and the assignments and the last step checks if all states are safe.

\subsubsection{Step 2:}
The check if the state is safe corresponding to each process runs in a loop and goes as follows:\\

We try to find an index \textit{i} such that both

\begin{itemize}
\item $Finish[i]== false$
\item $Need_i <= Work$
\end{itemize}

If we can't find such an \textit{i}, we continue to step 4, which means that the safety check will fail. If the check goes through, we go to the next step.

\subsubsection{Step 3:}
We now change work and set Finish for current i to true, like this:

\begin{itemize}
\item $Work = Work + Allocation_i$
\item $Finish[i] = true$
\end{itemize}

After this, we go to step 2 again for the next i in the loop.

\subsubsection{Step 4:}
When we've looped through step 2 and 3 until finished or until we jumped to step 4 at the end of step 2, we want to check if all the entries in the \textit{Finish} vector are set to true (here 1). If it is, the state is safe, else it isn't.\\

The function we've made which uses this algoritm returns 1 if the state is safe and 0 otherwise. This is then used when checking the initial state and also when processes are requesting resources to check if the state following the allocation is safe or else we backtrack.

\section{Testing}
The testing consists of a simulation, where we look at the provided data during the simulation and assure that it is corresponding to the expected behaviour.

The simulation program is made to print the available vector and the resource request vector in order for us to check the results. The testing made is almost purely informal black box testing. Of course we also went through some informal white box testing, finding the critical areas etc.\\

When the program is run, it will continue indefinately; acquirering and releasing resources to the system. This is alse what should happen in the simulation and therefore we see the tests so far as a success.\\