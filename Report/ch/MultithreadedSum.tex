\chapter{Multithreaded Sum}

\section{Requirements}

\begin{itemize}
\item Calculate the sum of the roots of numbers 0..N
\item Use multiple threads to do so
\end{itemize}

\section{Solution}
we have used a code snippet from our course book(Operating System Concepts 9th edition P. 171) which we based our solution on.

The major alteration is in the form of us adding a struct to contain information about each processes 'workload', this way we pass them in as the runners arguments and can access them in the sub processes.

\section{Testing}

\definecolor{gr1}{gray}{0.65}
\definecolor{gr2}{gray}{0.85}


\newcommand{\tblhead}[1] {
\multicolumn{2}{l}{#1}\\
\hline
}

\newcommand{\row}[2] {
\hline
\rowcolor{gr1}
Inputs&\\
\hline \hline
\rowcolor{gr2}
Calculations: & #1 \\
\hline
\rowcolor{gr2}
Threads: & #2 \\
\hline \hline
}

\newcommand{\botrow}[2] {
\hline
\rowcolor{gr1}
Inputs&\\
\hline \hline
\rowcolor{gr2}
Calculations: & #1 \\
\hline
\rowcolor{gr2}
Threads: & #2 \\
\hline
}

\begin{table}
\begin{tabular}{| p{\dimexpr.2\textwidth} | p{\dimexpr.5\textwidth-4\tabcolsep} |}
\tblhead{Results}
\row{42}{33}
\row{55}{666.99}
\botrow{8934}{76}
\end{tabular}
\end{table}